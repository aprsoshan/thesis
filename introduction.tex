\chapter{序論}

\section{背景}
人間の操作を必要としない自律移動ロボットに対して期待が高まっている.

自律移動ロボットの基本的な技術の1つに自己位置推定の技術がある.

自己位置推定の研究では,自律移動ロボットがセンサーなどからどのような情報が得られているのか、
またその情報から自己位置推定の結果であるパーティクルをどのように予測しているかを
Rvizなどのモニタリングツールをつかって観察する必要がある.
そのため,実験時に観察者は屈んだ姿勢でロボットの追跡をしつつ、


% 自己位置推定の研究では、自律移動ロボットがセンサーなどからどのような情報を取得しているか

\section{先行事例}

\subsection{AR ロボットコントローラ―}

\cite{鈴木勇矢2019ARロボットコントローラー}

\subsection{AR マーカーを用いた手法}

% dvipdfmxとhereのテスト
%\begin{figure}[H]
%	\begin{center}
%		\includegraphics[width=1.0\linewidth]{../zero.png}
%		\caption{}
%		\label{fig:}
%	\end{center}
%\end{figure}
%
